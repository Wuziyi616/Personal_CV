%%%%%%%%%%%%%%%%%%%%%%%%%%%%%%%%%%%%%%%%%
% Medium Length Professional CV
% LaTeX Template
% Version 2.0 (8/5/13)
%
% This template has been downloaded from:
% http://www.LaTeXTemplates.com
%
% Original author:
% Rishi Shah 
%
% Important note:
% This template requires the resume.cls file to be in the same directory as the
% .tex file. The resume.cls file provides the resume style used for structuring the
% document.
%
%%%%%%%%%%%%%%%%%%%%%%%%%%%%%%%%%%%%%%%%%

%----------------------------------------------------------------------------------------
%	PACKAGES AND OTHER DOCUMENT CONFIGURATIONS
%----------------------------------------------------------------------------------------

\documentclass{resume} % Use the custom resume.cls style

\usepackage[UTF8]{ctex}
\usepackage[left=0.7in,top=0.6in,right=0.7in,bottom=0.6in]{geometry} % Document margins
\newcommand{\tab}[1]{\hspace{.2667\textwidth}\rlap{#1}}
\newcommand{\itab}[1]{\hspace{0em}\rlap{#1}}
\name{吴紫屹} % Your name
\address{紫荆学生公寓二号楼 \\ 清华大学 \\ 中国\ 北京, 100084} % Your address
%\address{123 Pleasant Lane \\ City, State 12345} % Your secondary addess (optional)
\address{(+86) 18810237672 \\ \href{mailto:wuzy17@mails.tsinghua.edu.cn}{wuzy17@mails.tsinghua.edu.cn} \\ \href{mailto:dazitu616@gmail.com}{dazitu616@gmail.com}} % Your phone number and email

\begin{document}

%----------------------------------------------------------------------------------------
%	EDUCATION SECTION
%----------------------------------------------------------------------------------------

\begin{rSection}{教育背景}

{\textbf{清华大学}, 中国\ 北京} \hfill {\em 2017.8 -- 2021.7 (expected)}
\begin{itemize}
    信息学院\ 自动化系\ \textbf{工程学士学位} (expected)
    \item \textbf{GPA: 3.9/4.0, Ranking: 2$^{nd}$/173}
    %\item \textbf{本科 } 经管学院 (预期, 第二学位)
    \item 清华大学星火计划\ \textbf{理事}
\end{itemize}

\textbf{核心课程}
\begin{itemize}
    \item \textbf{数学}: 微积分 A (4.0/4.0), 
    线性代数 (4.0/4.0), 
    复变函数引论 (4.0/4.0), 
    概率论与数理统计 (4.0/4.0), 
    信号与系统分析 (4.0/4.0), 
    数值分析与算法 (4.0/4.0), 
    系统工程导论 (4.0/4.0), 
    运筹学 (4.0/4.0)
    \item \textbf{编程}: 计算机语言与程序设计 (4.0/4.0), 
    C++ 程序设计 (4.0/4.0), 
    数据结构与算法 (4.0/4.0), 
    计算机原理与应用 (4.0/4.0), 
    计算机网络与应用 (4.0/4.0), 
    人工智能基础 (4.0/4.0), 
    模式识别与机器学习 (4.0/4.0)\newline
\end{itemize}

\end{rSection}

\begin{rSection}{奖学金 \& 荣誉}

\begin{itemize}
    \item \textbf{2019 } \textbf{方崇智奖学金 } (清华大学自动化系最高等级荣誉, \textbf{0.1\%})
    \item \textbf{2019 } \textbf{星火计划成员 } (清华大学校内学术科研领域最为顶级的学生组织, \textbf{\textless \, 1\%})
    \item \textbf{2018 } \textbf{国家奖学金 } (中国政府授予的最高级别官方奖学金, \textbf{\textless \, 0.1\%})
    \item \textbf{2018 } 第二十届 \textbf{电子设计大赛冠军 } (清华大学校内在\textbf{电子设计}领域的最高级别赛事)
    \item \textbf{2017 } 第一届 \textbf{人工智能挑战赛A2组第五名 } (清华大学校内在\textbf{AI}领域的最高级别赛事)\newline
\end{itemize}

\end{rSection}
%--------------------------------------------------------------------------------
%    Projects And Seminars
%-----------------------------------------------------------------------------------------------
\begin{rSection}{出版 \& 投稿}

\begin{enumerate}
\item[1] Ziwei Wang, Jiwen Lu, \textbf{Ziyi Wu}, Jie Zhou. Learning Efficient Binarized Object Detectors with Information Compression. In submission to \textit{IEEE Transactions on Pattern Analysis and Machine Intelligence.} Under review.
%\item[1] Ziwei Wang, \textbf{Ziyi Wu}, Jiwen Lu, Jie Zhou. Instance Similarity Learning for Unsupervised Feature Representation. In submission to \textit{2020 European Conference on Computer Vision (ECCV).} Under review.
\item[2] Ziwei Wang, \textbf{Ziyi Wu}, Jiwen Lu, Jie Zhou. BiDet: An Efficient Binarized Object Detector. Accepted by \textit{2020 IEEE Conference on Computer Vision and Pattern Recognition (CVPR).} \href{https://arxiv.org/abs/2003.03961}{arXiv}
\item[3] Zimeng Tan, Yongjie Duan, \textbf{Ziyi Wu}, Jianjiang Feng, Jie Zhou. A Cascade Regression Model for Anatomical Landmark Detection. Accepted by \textit{2019 Medical Image Computing and Computer Assisted Intervention (MICCAI) Workshop.} \href{https://link.springer.com/chapter/10.1007/978-3-030-39074-7_5}{Springer}
\item[4] Zhanwei Xu, \textbf{Ziyi Wu}, Jianjiang Feng. CFUN: Combining Faster R-CNN and U-net Network for Efficient Whole Heart Segmentation. \href{https://arxiv.org/abs/1812.04914}{arXiv}\newline
\end{enumerate}

\end{rSection}

\begin{rSection}{科研兴趣}

\begin{description}
    \item[领域] \quad\quad\, 
    %Deep Hashing, Binary Representation, Binary Neural Networks, 
    高效推断, 
    %Model Compression, 
    3D 视觉, %Point Cloud, 
    无监督/自监督学习
    %Medical Image Processing
    %\item[方法] \quad\quad\, 深度学习, 强化学习, 神经网络, 信息论\newline
\end{description}

\end{rSection}

\begin{rSection}{科研经历}

{\textbf{Stanford University}, CA, U.S.} \hfill {\em May, 2020 -- present}\newline
\emph{\href{https://geometry.stanford.edu/}{Geometric Computing Group}, Department of Computer Science}\newline
Research Assistant, Advisors: Profs. \href{https://geometry.stanford.edu/member/guibas/index.html}{Leonidas Guibas}\newline
\textbf{Project: Adversarial Attack and Defense in 3D Point Clouds}
\begin{itemize}
    \item 提出现有攻击方法可以分为三大类,并以此为据分析了现有防御方法的不足之处
    \item 采用基于优化的防御方法,充分利用输入点云和几何重建的信息,实现对于三类攻击的防御
    \item 我们提出的方法在面对多种攻击、采用不同受攻击网络结构的情况下,均取得了最佳的防御效果\newline
\end{itemize}

{\textbf{Tsinghua University}, Beijing, China} \hfill {\em Apr, 2019 -- present}\newline
\emph{\href{http://ivg.au.tsinghua.edu.cn/index.php}{Intelligent Vision Group}, Department of Automation}\newline
Research Assistant, Advisors: Profs. \href{http://ivg.au.tsinghua.edu.cn/Jiwen_Lu/}{Jiwen Lu} \& \href{https://www.tsinghua.edu.cn/publish/auen/1713/2011/20110506105532098625469/20110506105532098625469_.html}{Jie Zhou}\newline
\textbf{Project: BiDet: An Efficient Binarized Object Detector}
\begin{itemize}
    \item 首次考虑在目标检测任务中引入二值网络,从而大大降低存储和计算开销
    \item 采用信息瓶颈理论来去除学习到特征中的冗余信息,从而充分利用二值网络的表征能力;同时学习稀疏先验,使得检测结果聚集于信息量丰富的区域来减少假阳性
    \item 我们提出的方法在多种检测网络、多个数据集上都取得了最佳的性能\newline
\end{itemize}

\textbf{Project: Learning Efficient Binarized Object Detectors with Information Compression}
\begin{itemize}
    \item 作为 BiDet 的期刊扩充版本,我们提出对不同输入样本自动调节信息瓶颈松紧,同时自适应学习稀疏先验,以此来更充分地去除假阳性
    \item 在多种检测网络、多个数据集上性能超越 BiDet,达到了新的 SOTA
    \item 我们将 AutoBiDet 策略泛化到了其他高效推断方法,如剪枝、量化上,均提升了对应方法的性能,证明了我们方法的普适性\newline
\end{itemize}

\end{rSection}
%----------------------------------------------------------------------------------------
%	TECHNICAL STRENGTHS SECTION
%----------------------------------------------------------------------------------------

\begin{rSection}{编程能力}

\begin{description}
    \item[熟练掌握] \quad\quad Python, Pytorch, C/C++, C\#, Markdown
    %\item[基本熟悉] \quad\quad Java, MATLAB, SQL, Linux, TensorFlow, Keras, etc
    \item[大致了解] \quad\quad Linux, TensorFlow, Keras, LaTeX, etc.
\end{description}

\end{rSection}

\begin{rSection}{语言能力}

\begin{description}
    \item[TOEFL iBT] \quad 109/120 \quad (Reading 30, Listening 26, Speaking 23, Writing 30)
    \item[GRE] \quad\quad\quad\quad\,\,\, 333/340+4.5/6.0 (Verbal 163, Quantitative 170, Analytical Writing 4.5)%\newline
\end{description}

\end{rSection}

\end{document}
