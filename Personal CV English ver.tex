%%%%%%%%%%%%%%%%%%%%%%%%%%%%%%%%%%%%%%%%%
% Medium Length Professional CV
% LaTeX Template
% Version 2.0 (8/5/13)
%
% This template has been downloaded from:
% http://www.LaTeXTemplates.com
%
% Original author:
% Rishi Shah 
%
% Important note:
% This template requires the resume.cls file to be in the same directory as the
% .tex file. The resume.cls file provides the resume style used for structuring the
% document.
%
%%%%%%%%%%%%%%%%%%%%%%%%%%%%%%%%%%%%%%%%%

%----------------------------------------------------------------------------------------
%	PACKAGES AND OTHER DOCUMENT CONFIGURATIONS
%----------------------------------------------------------------------------------------

\documentclass{resume} % Use the custom resume.cls style

\usepackage{setspace}
\usepackage[left=0.6in,top=0.5in,right=0.6in,bottom=0.5in]{geometry} % Document margins
\newcommand{\tab}[1]{\hspace{.2667\textwidth}\rlap{#1}}
\newcommand{\itab}[1]{\hspace{0em}\rlap{#1}}
\name{ZIYI WU} % Your name
\address{2\# Zijing Student Apartment \\ Tsinghua University \\ Beijing, 100084, P.R.China} % Your address
%\address{123 Pleasant Lane \\ City, State 12345} % Your secondary addess (optional)
\address{(+86) 18810237672 \\ \href{mailto:wuzy17@mails.tsinghua.edu.cn}{wuzy17@mails.tsinghua.edu.cn} \\ %\href{mailto:dazitu616@gmail.com}{dazitu616@gmail.com} \\
\url{https://wuziyi616.github.io/}}
%https://github.com/Wuziyi616} % Your phone number and email

\begin{document}

%----------------------------------------------------------------------------------------
%	EDUCATION SECTION
%----------------------------------------------------------------------------------------

\begin{rSection}{Education}

{\textbf{Tsinghua University}, Beijing, China} \hfill {\em Aug, 2017 -- Jul, 2021 (expected)}
\begin{itemize}
    \item \textbf{Bachelor} of Engineering in the Department of Automation (expected)
    \item \textbf{GPA: 3.9/4.0, Ranking: 2$^{nd}$/173}
    \item \textbf{Bachelor} of Management in the School of Economy and Management (expected, second degree)
    \item \textbf{Chairman} of Spark Program, Tsinghua University
    %\item \textbf{Captain} of the swimming team of the Department of Automation
\end{itemize}

\textbf{Core Courses}
\begin{spacing}{1.05}
\begin{itemize}
    \item \textbf{Mathematics}: Calculus A (4.0/4.0), 
    Linear Algebra (4.0/4.0), 
    Introduction to Complex Analysis (4.0/4.0), 
    Probability and Statistics (4.0/4.0), 
    %Signals and System Analysis (4.0/4.0), 
    %Numerical Analysis and Algorithms (4.0/4.0), 
    %Introduction to Systems Engineering (4.0/4.0), 
    Operations Research (4.0/4.0)
    \item \textbf{Programming}: Computer Languages and Programming (4.0/4.0), 
    C++ Program Design and Training (4.0/4.0), 
    Data Structure and Algorithms (4.0/4.0), 
    %Computer Principles and Applications (4.0/4.0), 
    %Computer Network and Applications (4.0/4.0), 
    Fundamental Artificial Intelligence (4.0/4.0), 
    Pattern Recognition and Machine Learning (4.0/4.0)
    %\end{spacing}
    %\newline
\end{itemize}
\end{spacing}

\end{rSection}

\begin{rSection}{Scholarships \& Awards}
\begin{spacing}{1.05}
\begin{itemize}
    \item \textbf{2020} \textbf{Xiaomi Scholarship} (Highest honor in Tsinghua University sponsored by Xiaomi Corp., \textbf{0.1\%})
    \item \textbf{2019} \textbf{Fang Chongzhi Scholarship} (Highest honor in the Dept. of Automation, \textbf{0.1\%})
    \item \textbf{2019} \textbf{Innovation Award of Science and Technology} (Awarded to undergraduate students with excellent scientific potential in Tsinghua University, \textbf{0.2\%})
    \item \textbf{2019} \textbf{Tsinghua Spark Program Membership} (Top student program in the field of academic research in Tsinghua University, \textbf{\textless~1\%})
    \item \textbf{2018} \textbf{National Scholarship} (Highest scholarship given by the Chinese government, \textbf{\textless~0.1\%})
    \item \textbf{2018} \textbf{Champion} in the 20$^{th}$ \textbf{Electronic Design Competition}, Tsinghua University% (Highest level competition in Tsinghua University in the field of \textbf{Electronic Engineering})
    \item \textbf{2018} \textbf{5$^{th}$ place} in the 1$^{st}$ \textbf{Artificial Intelligence Challenge}, Tsinghua University% Group A2 (Top level challenge in Tsinghua University in the field of \textbf{AI})
    %\newline
\end{itemize}
\end{spacing}

\end{rSection}
%--------------------------------------------------------------------------------
%    Projects And Seminars
%-----------------------------------------------------------------------------------------------
\begin{rSection}{Publications \& Manuscripts}

\begin{enumerate}
\item[1] \textbf{Ziyi Wu}$^{*}$, Yueqi Duan$^{*}$, He Wang, Qingnan Fan, Leonidas J. Guibas. IF-Defense: 3D Adversarial Point Cloud Defense via Implicit Function based Restoration. In submission to \textit{International Conference on Learning Representations (ICLR).} Under review.
\item[2] Ziwei Wang, Jiwen Lu, \textbf{Ziyi Wu}, Jie Zhou. Learning Efficient Binarized Object Detectors with Information Compression. In submission to \textit{IEEE Transactions on Pattern Analysis and Machine Intelligence (TPAMI).} Under major revision.
\item[3] Ziwei Wang, \textbf{Ziyi Wu}, Jiwen Lu, Jie Zhou. BiDet: An Efficient Binarized Object Detector. Accepted by \textit{2020 IEEE Conference on Computer Vision and Pattern Recognition (CVPR).} \href{https://arxiv.org/abs/2003.03961}{arXiv}
\item[4] Zimeng Tan, Yongjie Duan, \textbf{Ziyi Wu}, Jianjiang Feng, Jie Zhou. A Cascade Regression Model for Anatomical Landmark Detection. Accepted by \textit{2019 Medical Image Computing and Computer Assisted Intervention (MICCAI) Workshop.} \href{https://link.springer.com/chapter/10.1007/978-3-030-39074-7_5}{Springer}
%\item[4] Zhanwei Xu, \textbf{Ziyi Wu}, Jianjiang Feng. CFUN: Combining Faster R-CNN and U-net Network for Efficient Whole Heart Segmentation. \href{https://arxiv.org/abs/1812.04914}{arXiv}
%\newline
\end{enumerate}

\end{rSection}

\begin{rSection}{Research Interest}

\begin{description}
    \item[Fields] \quad\quad\, 
    %Deep Hashing, Binary Representation, Binary Neural Networks, 
    Efficient Inference, 
    %Model Compression, 
    3D Vision, %Point Cloud, 
    Unsupervised/Self-supervised Learning
    %Medical Image Processing
    \item[Methods] \quad Deep Learning, Reinforcement Learning, Neural Networks, Information Theory
    \newline
\end{description}

\end{rSection}

\begin{rSection}{Research Experiences}

{\textbf{Stanford University}, CA, U.S.} \hfill {\em May, 2020 -- present}\newline
\emph{\href{https://geometry.stanford.edu/}{Geometric Computing Group}, Department of Computer Science}\newline
Research Assistant, Advisors: Profs. \href{https://geometry.stanford.edu/member/guibas/index.html}{Leonidas Guibas}\newline
\textbf{Project: IF-Defense: 3D Adversarial Point Cloud Defense via Implicit Function based Restoration}
\begin{itemize}
    \item Summarize the effects of 3D adversarial attacks on point cloud into two aspects through comprehensive study of existing attack methods
    \item Propose a novel defense algorithm for 3D point cloud via accurate surface recovery and optimization based point restoration
    \item Achieve state-of-the-art defense performance against all existing attacks on four typical point cloud networks\newline
\end{itemize}

{\textbf{Tsinghua University}, Beijing, China} \hfill {\em Apr, 2019 -- Apr, 2020}\newline
\emph{\href{http://ivg.au.tsinghua.edu.cn/index.php}{Intelligent Vision Group}, Department of Automation}\newline
Research Assistant, Advisors: Profs. \href{http://ivg.au.tsinghua.edu.cn/Jiwen_Lu/}{Jiwen Lu} \& \href{https://www.tsinghua.edu.cn/publish/auen/1713/2011/20110506105532098625469/20110506105532098625469_.html}{Jie Zhou}\newline
\textbf{Project: BiDet: An Efficient Binarized Object Detector}
\begin{itemize}
    \item Apply binary neural networks (BNNs) in the object detection task for efficient inference, which is the first attempt to the best of our knowledge
    \item Employ the Information Bottleneck (IB) principle for redundancy removal to fully utilize the capacity of BNNs and learn sparse object priors to eliminate the false positives in the prediction output
    \item Achieve state-of-the-art performance under various detection frameworks on large scale datasets compared with existing binary detectors\newline
\end{itemize}

\textbf{Project: Learning Efficient Binarized Object Detectors with Information Compression}
\begin{itemize}
    \item Propose AutoBiDet, which is an extension of BiDet that automatically adjusts the information bottleneck trade-off and utilizes class-aware sparse priors to alleviate the false positives more effectively
    \item Achieve new state-of-the-art performance on large scale datasets comparing to BiDet
    \item Generalize AutoBiDet to boost the performance of other efficient inference algorithms including quantization, pruning and light-weight model design to show the universality of our proposed method\newline
\end{itemize}

\iffalse
{\textbf{Tsinghua University}, Beijing, China} \hfill {\em Aug, 2018 -- Apr, 2019}\newline
\emph{\href{http://ivg.au.tsinghua.edu.cn/index.php}{Intelligent Vision Group}, Department of Automation}\newline
Research Assistant, Advisors: Profs. \href{https://www.tsinghua.edu.cn/publish/auen/1713/2011/20110506103534247867775/20110506103534247867775_.html}{Jianjiang Feng} \& \href{https://www.tsinghua.edu.cn/publish/auen/1713/2011/20110506105532098625469/20110506105532098625469_.html}{Jie Zhou}\newline
\textbf{Project: A Cascade Regression Model for Cardiac Landmark Detection}
\begin{itemize}
    \item Apply cascade method to achieve state-of-the-art cardiac landmark detection result
    \item Take the advantages of two structures that utilize global information and local context to get accurate and fast regression output
    \item Lay the foundation for weakly-supervised heart segmentation approaches in the future\newline
\end{itemize}
\fi

\iffalse
\textbf{Project: CFUN: Combining Faster R-CNN and U-net Network for Efficient Whole Heart Segmentation}
\begin{itemize}
    \item Propose a novel network architecture combining state-of-the-art object detection framework Faster R-CNN and segmentation framework U-net
    \item Introduce an elaborate-designed loss called edge loss which can greatly boost the ability of segmentation
    \item Achieve relatively competitive performances while reducing the time needed for medical image processing compared to state-of-the-art methods
\end{itemize}
\fi
\end{rSection}
%----------------------------------------------------------------------------------------
%	TECHNICAL STRENGTHS SECTION
%----------------------------------------------------------------------------------------

\begin{rSection}{Programming Skills}

\begin{description}
    \item[Proficient] \quad\quad\quad\quad\,\,\,\,\, Python, Pytorch, C\#, Markdown, \LaTeX
    \item[Familiar] \quad\quad\quad\quad\quad\quad Linux, C/C++, TensorFlow, Keras, etc.
    \newline
    %\item[Basic Knowledge] \quad\, MATLAB, Java, JavaScript, HTML/CSS, etc\newline
\end{description}

\end{rSection}

\begin{rSection}{Language Skills}

\begin{description}
    \item[TOEFL iBT] \quad 109/120 \quad (Reading 30, Listening 26, Speaking 23, Writing 30)
    \item[GRE] \quad\quad\quad\quad\,\,\, 333/340+4.5/6.0 \, (Verbal 163, Quantitative 170, Analytical Writing 4.5)
    %\newline
\end{description}

\end{rSection}

\end{document}
