%%%%%%%%%%%%%%%%%%%%%%%%%%%%%%%%%%%%%%%%%
% Medium Length Professional CV
% LaTeX Template
% Version 2.0 (8/5/13)
%
% This template has been downloaded from:
% http://www.LaTeXTemplates.com
%
% Original author:
% Rishi Shah 
%
% Important note:
% This template requires the resume.cls file to be in the same directory as the
% .tex file. The resume.cls file provides the resume style used for structuring the
% document.
%
%%%%%%%%%%%%%%%%%%%%%%%%%%%%%%%%%%%%%%%%%

%----------------------------------------------------------------------------------------
%	PACKAGES AND OTHER DOCUMENT CONFIGURATIONS
%----------------------------------------------------------------------------------------

\documentclass{resume} % Use the custom resume.cls style

%\usepackage{ulem}
\usepackage{setspace}
\usepackage[left=0.65in,top=0.55in,right=0.65in,bottom=0.65in]{geometry} % Document margins
\newcommand{\tab}[1]{\hspace{.2667\textwidth}\rlap{#1}}
\newcommand{\itab}[1]{\hspace{0em}\rlap{#1}}
\name{ZIYI WU} % Your name
\address{
%2\# Zijing Student Apartment \\ 
Tsinghua University \\ 
Beijing, 100084, P.R.China
} % Your address
%\address{123 Pleasant Lane \\ City, State 12345} % Your secondary addess (optional)
\address{
(+86) 188 1023 7672 \\ 
% \href{mailto:wuzy17@mails.tsinghua.edu.cn}{wuzy17@mails.tsinghua.edu.cn} \\ 
\href{mailto:dazitu616@gmail.com}{dazitu616@gmail.com} \\ 
\url{https://wuziyi616.github.io/}}
%https://github.com/Wuziyi616} % Your phone number and email

\begin{document}

%----------------------------------------------------------------------------------------
%	EDUCATION SECTION
%----------------------------------------------------------------------------------------

\begin{rSection}{Education}

{\textbf{Tsinghua University}}, {Beijing, P.R.China}

\textbf{Bachelor} of Engineering in Automation \hfill {\em Aug, 2017 -- Jul, 2021}
\begin{itemize}
    %\item \textbf{Bachelor} of Engineering in the Department of Automation (expected)
    \item \textbf{GPA: 3.9/4.0, Ranking: 2$^{nd}$/173}
    %\item \textbf{Bachelor} of Management in the School of Economy and Management (expected, second degree)
    \item \textbf{Chairman} of Spark Program, Tsinghua University
    %\item \textbf{Captain} of the swimming team of the Department of Automation
\end{itemize}

\textbf{Core Courses}
\begin{spacing}{1.05}
\begin{itemize}
    \item \textbf{Mathematics}: Calculus A (4.0/4.0), 
    Linear Algebra (4.0/4.0), 
    Introduction to Complex Analysis (4.0/4.0), 
    Probability and Statistics (4.0/4.0), 
    %Signals and System Analysis (4.0/4.0), 
    %Numerical Analysis and Algorithms (4.0/4.0), 
    %Introduction to Systems Engineering (4.0/4.0), 
    Operations Research (4.0/4.0)
    \item \textbf{CS}: Computer Languages and Programming (4.0/4.0), 
    C++ Program Design and Training (4.0/4.0), 
    Data Structure and Algorithms (4.0/4.0), 
    %Computer Principles and Applications (4.0/4.0), 
    Computer Network and Applications (4.0/4.0), 
    Fundamental Artificial Intelligence (4.0/4.0), 
    Pattern Recognition and Machine Learning (4.0/4.0)
    %\end{spacing}
    %\newline
\end{itemize}
\end{spacing}

\end{rSection}

\vspace{-0.3cm}

\begin{rSection}{Scholarships \& Awards}
%\begin{spacing}{1.05}
\begin{itemize}
    \item \textbf{2020} \textbf{SenseTime Scholarship} (Awarded to only 21 undergraduate AI researchers in P.R.China)
    \item \textbf{2020} \textbf{Xiaomi Scholarship} (Highest scholarship in Tsinghua sponsored by Xiaomi Corp., \textbf{0.1\%})
    \item \textbf{2020, 2019} \textbf{Tsinghua Innovation Award of Science and Technology} (\textbf{0.2\%})
    \item \textbf{2019} \textbf{Fang Chongzhi Scholarship} (Highest honor in the Dept. of Automation, \textbf{0.1\%})
    \item \textbf{2019} \textbf{Tsinghua Spark Program Membership} (Top student program in academic research, \textbf{1\%})
    \item \textbf{2018} \textbf{National Scholarship} (Highest scholarship awarded by the Chinese government, \textbf{\textless~0.1\%})
    \item \textbf{2018} \textbf{Champion} in the 20$^{th}$ \textbf{Electronic Design Competition}, Tsinghua University (\textbf{1/120})% (Highest level competition in Tsinghua University in the field of \textbf{Electronic Engineering})
    \item \textbf{2018} \textbf{5$^{th}$ place} in the 1$^{st}$ \textbf{Artificial Intelligence Challenge}, Tsinghua University (\textbf{5/150})% Group A2 (Top level challenge in Tsinghua University in the field of \textbf{AI})
    \newline
\end{itemize}
%\end{spacing}

\end{rSection}

\vspace{-0.2cm}

\begin{rSection}{Publications \& Manuscripts}

\begin{enumerate}
\item[1] \textbf{Ziyi Wu}$^{*}$, Yueqi Duan$^{*}$, He Wang, Qingnan Fan, Leonidas J. Guibas. IF-Defense: 3D Adversarial Point Cloud Defense via Implicit Function based Restoration. Submitted to \textit{2021 IEEE/CVF Conference on Computer Vision and Pattern Recognition (CVPR)}. Under review. \href{https://arxiv.org/abs/2010.05272}{[\underline{arXiv}]}
\item[2] Ziwei Wang, Jiwen Lu, \textbf{Ziyi Wu}, Jie Zhou. Learning Efficient Binarized Object Detectors with Information Compression. Accepted by \textit{IEEE Transactions on Pattern Analysis and Machine Intelligence (T-PAMI)}.
\item[3] Ziwei Wang, \textbf{Ziyi Wu}, Jiwen Lu, Jie Zhou. BiDet: An Efficient Binarized Object Detector. Accepted by \textit{2020 IEEE/CVF Conference on Computer Vision and Pattern Recognition (CVPR)}. \href{https://arxiv.org/abs/2003.03961}{[\underline{arXiv}]}
\item[4] Zimeng Tan, Yongjie Duan, \textbf{Ziyi Wu}, Jianjiang Feng, Jie Zhou. A Cascade Regression Model for Anatomical Landmark Detection. Accepted by \textit{2019 Medical Image Computing and Computer Assisted Intervention (MICCAI) Workshop}. \href{https://link.springer.com/chapter/10.1007/978-3-030-39074-7_5}{[\underline{Springer}]}
\item[5] Zhanwei Xu, \textbf{Ziyi Wu}, Jianjiang Feng. CFUN: Combining Faster R-CNN and U-net Network for Efficient Whole Heart Segmentation. \href{https://arxiv.org/abs/1812.04914}{[\underline{arXiv}]}
%\newline
\end{enumerate}

\end{rSection}

%\clearpage

\begin{rSection}{Research Interest}

\begin{description}
    \item[Fields] \quad\quad\, 
    %Deep Hashing, Binary Representation, Binary Neural Networks, 
    %Model Compression, 
    3D Vision, 
    Video Analysis, 
    %Point Cloud Analysis, 
    Efficient Inference, 
    Unsupervised Learning
    %Medical Image Processing
    \item[Methods] \quad Deep Learning, 
    Reinforcement Learning, 
    Neural Networks, 
    Information Theory
    \newline
\end{description}

\end{rSection}

\vspace{-0.2cm}

\begin{rSection}{Research Experience}

{\textbf{Stanford University}}, {CA, U.S.}\newline
\emph{\href{https://geometry.stanford.edu/}{Geometric Computing Group}, Department of Computer Science} \hfill {\em May, 2020 -- Nov, 2020}\newline
Research Assistant, Advisor: Prof. \href{https://geometry.stanford.edu/member/guibas/index.html}{Leonidas Guibas}\newline
\textbf{Project: IF-Defense: 3D Adversarial Point Cloud Defense via Implicit Function based Restoration}
\begin{itemize}
    \item Summarized the effects of 3D adversarial attacks on point cloud into two aspects from a geometric perspective through comprehensive review of existing attack methods
    \item Proposed a novel defense algorithm for 3D point cloud which can simultaneously address the two attack effects via accurate surface recovery and optimization based point restoration
    \item Achieved state-of-the-art robustness against all existing attacks on five typical point cloud networks\newline
\end{itemize}

\vspace{-0.15cm}

{\textbf{Tsinghua University}}, {Beijing, P.R.China}\newline
\emph{\href{http://ivg.au.tsinghua.edu.cn/index.php}{Intelligent Vision Group}, Department of Automation} \hfill {\em Apr, 2019 -- Apr, 2020}\newline
Research Assistant, Advisors: Profs. \href{http://ivg.au.tsinghua.edu.cn/Jiwen_Lu/}{Jiwen Lu} \& \href{https://www.tsinghua.edu.cn/publish/auen/1713/2011/20110506105532098625469/20110506105532098625469_.html}{Jie Zhou}\newline
\textbf{Project: BiDet: An Efficient Binarized Object Detector}
\begin{itemize}
    \item Applied binary neural networks (BNNs) in the object detection task to reduce storage and computational cost, which was the first attempt to the best of our knowledge
    \item Employed the Information Bottleneck (IB) method for redundancy removal to fully utilize the capacity of BNNs and learned sparse object priors to eliminate the false positives in prediction outputs
    \item Boosted the performance significantly for both one-stage and two-stage detectors while reducing the model size and inference time by more than $10\times$\newline
\end{itemize}

\vspace{-0.35cm}

\textbf{Project: Learning Efficient Binarized Object Detectors with Information Compression}
\begin{itemize}
    \item Proposed AutoBiDet, an extension of BiDet that automatically adjusts the IB trade-off and utilizes class-aware sparse object priors to alleviate the false positives more effectively
    \item Outperformed BiDet by a sizeable margin on both PASCAL VOC and MS COCO datasets when combined with more backbones and detection frameworks
    \item Generalized the techniques used in AutoBiDet to improve other model compression methods including low-bit quantization and channel pruning to show the universality of our approach\newline
\end{itemize}

\iffalse
{\textbf{Tsinghua University}, Beijing, China} \hfill {\em Aug, 2018 -- Apr, 2019}\newline
\emph{\href{http://ivg.au.tsinghua.edu.cn/index.php}{Intelligent Vision Group}, Department of Automation}\newline
Research Assistant, Advisors: Profs. \href{https://www.tsinghua.edu.cn/publish/auen/1713/2011/20110506103534247867775/20110506103534247867775_.html}{Jianjiang Feng} \& \href{https://www.tsinghua.edu.cn/publish/auen/1713/2011/20110506105532098625469/20110506105532098625469_.html}{Jie Zhou}\newline
\textbf{Project: A Cascade Regression Model for Cardiac Landmark Detection}
\begin{itemize}
    \item Apply cascade method to achieve state-of-the-art cardiac landmark detection result
    \item Take the advantages of two structures that utilize global information and local context to get accurate and fast regression output
    \item Lay the foundation for weakly-supervised heart segmentation approaches in the future\newline
\end{itemize}
\fi

\iffalse
\textbf{Project: CFUN: Combining Faster R-CNN and U-net Network for Efficient Whole Heart Segmentation}
\begin{itemize}
    \item Propose a novel network architecture combining state-of-the-art object detection framework Faster R-CNN and segmentation framework U-net
    \item Introduce an elaborate-designed loss called edge loss which can greatly boost the ability of segmentation
    \item Achieve relatively competitive performances while reducing the time needed for medical image processing compared to state-of-the-art methods
\end{itemize}
\fi
\end{rSection}

\vspace{-0.2cm}

\begin{rSection}{Programming Skills}

\begin{description}
    \item[Proficient] \quad\,\,\,\,\, Python, PyTorch, C\#, Markdown, LaTeX, Git
    \item[Familiar] \quad\quad\quad Linux, C/C++, TensorFlow, Keras, MATLAB, HTML, etc.
    \newline
    %\item[Basic Knowledge] \quad\, MATLAB, Java, JavaScript, HTML/CSS, etc\newline
\end{description}

\end{rSection}

\vspace{-0.2cm}

\begin{rSection}{Language Skills}

\begin{description}
    \item[TOEFL iBT] \quad 109/120 (Reading 30, Listening 26, Speaking 23, Writing 30)
    \item[GRE] \quad\quad\quad\quad\,\,\, 333/340+4.5/6.0 (Verbal 163, Quantitative 170, Analytical Writing 4.5)
    %\newline
\end{description}

\end{rSection}

\end{document}
