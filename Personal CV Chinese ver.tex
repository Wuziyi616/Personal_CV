%%%%%%%%%%%%%%%%%%%%%%%%%%%%%%%%%%%%%%%%%
% Medium Length Professional CV
% LaTeX Template
% Version 2.0 (8/5/13)
%
% This template has been downloaded from:
% http://www.LaTeXTemplates.com
%
% Original author:
% Rishi Shah 
%
% Important note:
% This template requires the resume.cls file to be in the same directory as the
% .tex file. The resume.cls file provides the resume style used for structuring the
% document.
%
%%%%%%%%%%%%%%%%%%%%%%%%%%%%%%%%%%%%%%%%%

%----------------------------------------------------------------------------------------
%	PACKAGES AND OTHER DOCUMENT CONFIGURATIONS
%----------------------------------------------------------------------------------------

\documentclass{resume} % Use the custom resume.cls style

\usepackage[UTF8]{ctex}
\usepackage[left=0.6in,top=0.5in,right=0.6in,bottom=0.5in]{geometry} % Document margins
\newcommand{\tab}[1]{\hspace{.2667\textwidth}\rlap{#1}}
\newcommand{\itab}[1]{\hspace{0em}\rlap{#1}}
\name{吴紫屹} % Your name
\address{紫荆学生公寓二号楼 \\ 清华大学 \\ 中国\ 北京, 100084} % Your address
%\address{123 Pleasant Lane \\ City, State 12345} % Your secondary addess (optional)
\address{(+86) 18810237672 \\ \href{mailto:wuzy17@mails.tsinghua.edu.cn}{wuzy17@mails.tsinghua.edu.cn} \\ %\href{mailto:dazitu616@gmail.com}{dazitu616@gmail.com} \\
\url{https://wuziyi616.github.io/}}
%https://github.com/Wuziyi616} % Your phone number and email

\begin{document}

%----------------------------------------------------------------------------------------
%	EDUCATION SECTION
%----------------------------------------------------------------------------------------

\begin{rSection}{教育背景}

{\textbf{清华大学}, 中国\ 北京} \hfill {\em 2017.8 -- 2021.7 (预计)}
\begin{itemize}
    信息学院\ 自动化系\ \textbf{工程学士学位} (预计)
    \item \textbf{GPA: 3.9/4.0, Ranking: 2$^{nd}$/173}
    %\item \textbf{本科 } 经管学院 (预计, 第二学位)
    %\item 清华大学星火计划\ \textbf{理事}
\end{itemize}

\textbf{核心课程}
\begin{itemize}
    \item \textbf{数学}: 微积分 A (4.0/4.0), 
    线性代数 (4.0/4.0), 
    复变函数引论 (4.0/4.0), 
    概率论与数理统计 (4.0/4.0), 
    信号与系统分析 (4.0/4.0), 
    数值分析与算法 (4.0/4.0), 
    系统工程导论 (4.0/4.0), 
    运筹学 (4.0/4.0)
    \item \textbf{编程}: 计算机语言与程序设计 (4.0/4.0), 
    C++ 程序设计 (4.0/4.0), 
    数据结构与算法 (4.0/4.0), 
    计算机原理与应用 (4.0/4.0), 
    计算机网络与应用 (4.0/4.0), 
    人工智能基础 (4.0/4.0), 
    模式识别与机器学习 (4.0/4.0)\newline
\end{itemize}

\end{rSection}

\begin{rSection}{奖学金 \& 荣誉}

\begin{itemize}
    \item \textbf{2020 } \textbf{小米奖学金 } (清华大学本科生中由企业资助的最高奖学金, \textbf{0.1\%})
    \item \textbf{2019 } \textbf{方崇智奖学金 } (清华大学自动化系最高等级荣誉, \textbf{0.1\%})
    \item \textbf{2019 } \textbf{清华大学科技创新奖 } (奖励给清华大学最有科研潜力的本科生, \textbf{0.2\%})
    \item \textbf{2019 } \textbf{星火计划成员 } (清华大学校内学术科研领域最为顶级的学生组织, \textbf{\textless \, 1\%})
    \item \textbf{2018 } \textbf{国家奖学金 } (中国政府授予的最高级别官方奖学金, \textbf{\textless \, 0.1\%})
    \item \textbf{2018 } 第二十届 \textbf{电子设计大赛冠军 } (清华大学校内在\textbf{电子设计}领域的最高级别赛事)
    \item \textbf{2017 } 第一届 \textbf{人工智能挑战赛 A2 组第五名 } (清华大学校内在 \textbf{AI} 领域的最高级别赛事)\newline
\end{itemize}

\end{rSection}
%--------------------------------------------------------------------------------
%    Projects And Seminars
%-----------------------------------------------------------------------------------------------
\begin{rSection}{出版 \& 投稿}

\begin{enumerate}
\item[1] \textbf{Ziyi Wu}$^{*}$, Yueqi Duan$^{*}$, He Wang, Qingnan Fan, Leonidas J. Guibas. IF-Defense: 3D Adversarial Point Cloud Defense via Implicit Function based Restoration. In submission to \textit{International Conference on Learning Representations (ICLR).} Under review.
\item[2] Ziwei Wang, Jiwen Lu, \textbf{Ziyi Wu}, Jie Zhou. Learning Efficient Binarized Object Detectors with Information Compression. In submission to \textit{IEEE Transactions on Pattern Analysis and Machine Intelligence.} Under major revision.
\item[3] Ziwei Wang, \textbf{Ziyi Wu}, Jiwen Lu, Jie Zhou. BiDet: An Efficient Binarized Object Detector. Accepted by \textit{2020 IEEE Conference on Computer Vision and Pattern Recognition (CVPR).} \href{https://arxiv.org/abs/2003.03961}{arXiv}
\item[4] Zimeng Tan, Yongjie Duan, \textbf{Ziyi Wu}, Jianjiang Feng, Jie Zhou. A Cascade Regression Model for Anatomical Landmark Detection. Accepted by \textit{2019 Medical Image Computing and Computer Assisted Intervention (MICCAI) Workshop.} \href{https://link.springer.com/chapter/10.1007/978-3-030-39074-7_5}{Springer}
%\item[4] Zhanwei Xu, \textbf{Ziyi Wu}, Jianjiang Feng. CFUN: Combining Faster R-CNN and U-net Network for Efficient Whole Heart Segmentation. \href{https://arxiv.org/abs/1812.04914}{arXiv}\newline
\end{enumerate}

\end{rSection}

\begin{rSection}{科研兴趣}

\begin{description}
    \item[领域] \quad\quad\, 
    %Deep Hashing, Binary Representation, Binary Neural Networks, 
    高效推断, 
    %Model Compression, 
    3D 视觉, %Point Cloud, 
    无监督/自监督学习, 
    信息论
    %Medical Image Processing
    %\item[方法] \quad\quad\, 深度学习, 强化学习, 神经网络, 信息论\newline
\end{description}

\end{rSection}

\begin{rSection}{科研经历}

{\textbf{Stanford University}, CA, U.S.} \hfill {\em May, 2020 -- present}\newline
\emph{\href{https://geometry.stanford.edu/}{Geometric Computing Group}, Department of Computer Science}\newline
Research Assistant, Advisors: Profs. \href{https://geometry.stanford.edu/member/guibas/index.html}{Leonidas Guibas}\newline
\textbf{Project: IF-Defense: 3D Adversarial Point Cloud Defense via Implicit Function based Restoration}
\begin{itemize}
    \item 通过全面调研,我们将现有点云攻击方法产生的效果分成了两大类,并基于此分析了前人防御方法的不足
    \item 提出一种基于精确表面恢复和坐标优化的新颖点云防御方法
    \item 在四种典型点云网络结构上取得了对抗全部现有点云攻击方法的最佳防御性能\newline
\end{itemize}

{\textbf{Tsinghua University}, Beijing, China} \hfill {\em Apr, 2019 -- present}\newline
\emph{\href{http://ivg.au.tsinghua.edu.cn/index.php}{Intelligent Vision Group}, Department of Automation}\newline
Research Assistant, Advisors: Profs. \href{http://ivg.au.tsinghua.edu.cn/Jiwen_Lu/}{Jiwen Lu} \& \href{https://www.tsinghua.edu.cn/publish/auen/1713/2011/20110506105532098625469/20110506105532098625469_.html}{Jie Zhou}\newline
\textbf{Project: BiDet: An Efficient Binarized Object Detector}
\begin{itemize}
    \item 首次考虑在目标检测任务中引入二值网络,从而大大降低存储和计算开销
    \item 采用信息瓶颈理论来去除学习到特征中的冗余信息,从而充分利用二值网络的表征能力;同时学习稀疏先验,使得检测结果聚集于信息量丰富的区域来减少假阳性
    \item 我们提出的方法在多种检测网络、多个数据集上都取得了最佳的性能\newline
\end{itemize}

\textbf{Project: Learning Efficient Binarized Object Detectors with Information Compression}
\begin{itemize}
    \item 作为 BiDet 的期刊扩充版本,我们提出对不同输入样本自动调节信息瓶颈松紧,同时自适应学习稀疏先验,以此来更充分地去除假阳性
    \item 在多种检测网络、多个数据集上性能超越 BiDet,达到了新的 SOTA
    \item 我们将 AutoBiDet 策略泛化到了其他高效推断方法,如剪枝、量化上,均提升了对应方法的性能,证明了我们方法的普适性\newline
\end{itemize}

\end{rSection}
%----------------------------------------------------------------------------------------
%	TECHNICAL STRENGTHS SECTION
%----------------------------------------------------------------------------------------

\begin{rSection}{编程能力}

\begin{description}
    \item[熟练掌握] \quad\quad Python, PyTorch, C\#, Markdown, \LaTeX
    %\item[基本熟悉] \quad\quad Java, MATLAB, SQL, Linux, TensorFlow, Keras, etc
    \item[基本了解] \quad\quad Linux, C/C++, TensorFlow, Keras, etc.
\end{description}

\end{rSection}

\begin{rSection}{语言能力}

\begin{description}
    \item[TOEFL iBT] \quad 109/120 \quad (Reading 30, Listening 26, Speaking 23, Writing 30)
    \item[GRE] \quad\quad\quad\quad\,\,\, 333/340+4.5/6.0 (Verbal 163, Quantitative 170, Analytical Writing 4.5)%\newline
\end{description}

\end{rSection}

\end{document}
